\documentclass{article}
\usepackage{amsmath, amsthm, amssymb}
\begin{document}
\pagestyle{headings}
\title{Copay: A Decentralized Multisignature Wallet}
\author{
Ryan X. Charles,
Matias Alejo Garcia,
Manuel Araoz,\\
Gustavo Cortez,
Mario Colque,
Juan Sosa
}
\date{}
\maketitle
\begin{abstract}
Bitcoin security can be enhanced by requiring multiple signatures for transactions (multisig), but all existing multisig solutions suffer from being either unfriendly to use or from requiring centralized key-signing servers that might fail.
Copay is designed to bring all of the security of multisig to a wallet that is as easy to use as an ordinary bitcoin wallet, while eliminating the possibility that the failure of a central server will deny service to users.
First, Copay introduces the notion of an M-of-N multisig wallet, where N particular people (or devices) join a wallet where every outbound transaction requires that M of them sign.
Communications between the copayers are p2p and wallets are stored client-side, so no central servers are required (servers are required for facilitating the p2p connections and interacting with the blockchain, but these are open-source and can be run by the users).
It runs on every platform, including platforms hostile to bitcoin.
Second, for this to be possible, we make use of many modern technologies:
1) It is written in javascript and HTML so that it runs on every platform, including phones,
2) We use p2p WebRTC connections between users so that communications have no dependence on central servers,
3) We use HTML 5 local storage to store the encrypted wallet so that storage has no dependence on central servers,
4) We use P2SH addresses so that sending to the wallet is as easy as sending to any bitcoin wallet,
5) We use BIP32 hierarchical deterministic keys so that backing up your wallet needs to be done only once, and generating new addresses can be done offline.
\end{abstract}

\section{Introduction}

Although the bitcoin protocol itself is cryptographically secure, individual users or companies may not be secure.
Normal pubkeyhash transactions require only one signature, which in turn requires only one private key.
If this private key or if the entire wallet of private keys is revealed to an attacker due to weak security from the user, then the user's bitcoins can be stolen.

Multisignature transactions offer a solution to this problem by requiring multiple signatures per transaction.
Each signature requires a different private key, which can all be stored on different devices in different places.
This reduces the probability that the bitcoins can be stolen, since not just one, but multiple devices must simultaneously be compromised.

Multisignature transactions have been possible in bitcoin since the genesis block with the OP\_CHECKMULTISIG operation. Multisignature transactions became standard on [TODO: insert date], meaning they are relayed by default by the reference client.
In order to make advanced transactions more accessible, pay-to-script-hash (P2SH) transactions were introduced on [TODO: insert date].
P2SH transactions allow a serialized script, the redeemScript, to be inserted into the scriptSig which is executed after the scriptPubKey.
The redeemScript can be any one of the standard scriptPubKeys, including a multisignature scriptPubKey.
In effect this means it is possible to pay to an address that requires multiple signatures to spend.

P2SH multisignature transactions have had an RPC interface in the reference client since [TODO: insert date].
However, the RPC interface is too technically challenging for most users, and is cumbersome even for technically sophisticated users.
The primary goal of Copay is to make the security of multisig accessible to ordinary users by having a friendly UI.


\section{Concepts}

Paragraph.

\section{Technologies}

Paragraph.

\section{Next Steps}

Paragraph.

\section{Conclusion}

Paragraph.

\end{document}
