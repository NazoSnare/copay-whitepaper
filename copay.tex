\documentclass{article}
\usepackage{amsmath, amsthm, amssymb}
\begin{document}
\pagestyle{headings}
\title{Copay: A Decentralized Multisignature Wallet}
\author{
Ryan X. Charles,
Matias Alejo Garcia,
Manuel Araoz,\\
Gustavo Cortez,
Mario Colque,
Juan Sosa
}
\date{}
\maketitle
\begin{abstract}
Bitcoin security can be enhanced by requiring multiple signatures for transactions (multisig), but all existing multisig solutions suffer from being either unfriendly to use or from requiring a centralized key-signing server that might fail.
Copay is designed to bring all of the security of multisig to a wallet that is as easy to use as an ordinary bitcoin wallet, while eliminating the possibility that the failure of a central server will deny service to users.
First, Copay introduces the notion of a general M-of-N multisig wallet, where N particular people (or devices) join a wallet where every outbound transaction requires that M of them sign.
Communications between the copayers are p2p and wallets are stored client-side, so no central servers are required (servers are required for facilitating the p2p connections and interacting with the blockchain, but these are open-source and can be run by the users).
It runs on every platform, including platforms hostile to bitcoin.
Second, for this to be possible, we make use of many modern technologies:
1) It is written in javascript and HTML so that it runs on every platform, including phones,
2) We use p2p WebRTC connections between users so that communications have no dependence on central servers,
3) We use HTML 5 local storage to store the encrypted wallet so that storage has no dependence on central servers,
4) We use P2SH addresses so that sending to the wallet is as easy as sending to any bitcoin wallet,
5) We use BIP32 hierarchical deterministic keys so that backing up your wallet needs to be done only once, and generating new addresses can be done offline.
\end{abstract}

\section{Introduction}

Paragraph.

\section{Middle}

Paragraph.

\section{Conclusion}

Schrodinger equation: $\psi(t)=e^{-iHt}\psi(0)$.

\end{document}
